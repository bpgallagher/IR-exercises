\exercise

\begin{enumerate}

  \item Define the TF-IDF$(w, d)$ weight for a word $w$ in a document $d$;

  \item Given the four texts:
  %
  \begin{enumerate}
    \item $T_1$ = \emph{``la bella casa''}
    \item $T_2$ = \emph{``le belle cose''}
    \item $T_3$ = \emph{``casa bella bella''}
    \item $T_4$ = \emph{``la casa le cose bella rosa''}
  \end{enumerate}
  %
  Show the inverted list with its dictionary and postings, compute the vectors
  TF-IDF of the four texts. Indicate the text which is more similar to $T_3$
  under the dot-product distance.

  \item Show how the TF-IDF matrix is space-efficiently stored in the posting
  lists.

\end{enumerate}

\solution
\label{10_text_ranking:tfid}

\begin{enumerate}

  \item The TF-IDF$(w, d)$ weight is defined as $$\text{TF-IDF}(w, d) =
  \text{TF}(w, d) \times \text{IDF}(w)$$ where $\text{TF}(w, d)$ is the number
  of occurrences of the word $w$ in the document $d$, while $\text{IDF}(w)$ is
  defined as $$\text{IDF}(w) = \log \frac{N}{\text{DF}(w)}$$ where $N$ is the
  total number of documents in our collection and $\text{DF}(w)$ is the number
  of documents containing the word $w$.

  \item We first construct the posting lists:
  %
  \begin{longtable}{rcl}
    \emph{bella} & $\rightarrow$ & $T_1$, $T_3$, $T_4$ \\
    \emph{belle} & $\rightarrow$ & $T_2$ \\
    \emph{casa} & $\rightarrow$ & $T_1$, $T_3$, $T_4$ \\
    \emph{cose} & $\rightarrow$ & $T_2$, $T_4$ \\
    \emph{la} & $\rightarrow$ & $T_1$, $T_4$ \\
    \emph{le} & $\rightarrow$ & $T_2$, $T_4$ \\
    \emph{rosa} & $\rightarrow$ & $T_4$ \\
  \end{longtable}
  %
  Then we apply the TF-IDF formula for all terms and all documents:
  %
  \begin{center}
    \begin{tabular}{c|c|c|c|c|}
                   & $T_1$ & $T_2$ & $T_3$ & $T_4$ \\ \hline &&&&\\[-1em]
      \emph{bella} & $1 \cdot\log\frac{4}{3}$ & 0
                   & $2 \cdot\log\frac{4}{3}$ & $1 \cdot\log\frac{4}{3}$\\[3pt]
      \emph{belle} & 0 & $1 \cdot\log\frac{4}{1}$ & 0 & 0 \\[3pt]
      \emph{casa}  & $1 \cdot\log\frac{4}{3}$ & 0
                   & $1 \cdot\log\frac{4}{3}$ & $1 \cdot\log\frac{4}{3}$ \\[3pt]
      \emph{cose}  & 0 & $1 \cdot\log\frac{4}{2}$
                   & 0 & $1 \cdot\log\frac{4}{2}$ \\[3pt]
      \emph{la}    & $1 \cdot\log\frac{4}{2}$ & 0
                   & 0 & $1 \cdot\log\frac{4}{2}$ \\[3pt]
      \emph{le}    & 0 & $1 \cdot\log\frac{4}{2}$
                   & 0 & $1 \cdot\log\frac{4}{2}$ \\[3pt]
      \emph{rosa}  & 0 & 0 & 0 & $1 \cdot\log\frac{4}{1}$ \\[3pt] \hline
    \end{tabular}
  \end{center}

  Since all but the first and the third element of the $T_3$ column are 0, the
  documents that are more similar to $T_3$ are the ones that have the highest
  values on that positions. In this case, we have a tie with $T_1$ and $T_4$,
  since their dot product have the same value: $$ T_3 \cdot T_1 = T_3 \cdot T_4
  = 3\log^2\frac{4}{3}.$$

  \item For every word $w$ we have in memory the value of $\text{DF}(w)$ (the
  length of its posting list), so the value of $\text{IDF}(w)$ is implicitly
  available. For every document $d$ in the posting list of the word $w$,
  instead, we can store its frequency $\text{TF}(w, d)$, which is typically
  small and thus can be stored in unary or gamma coding.

\end{enumerate}
