\exercise

Given the two adjacency lists of a Web graph for the nodes 14 and 15, namely

\begin{align*}
	14 &\rightarrow 3, 10, 11, 13, 14, 17, 19, 21, 25 \\
	15 &\rightarrow 5, 10, 11, 12, 14, 17, 19, 20, 21, 24, 33
\end{align*}

show how the list of "15" can be compressed given the list of "14" by means of
the algorithm adopted in WebGraph via copy-lists and copy-blocks.

\solution

The successors of "15" share the nodes of the posting list of "14" at positions 2,
3, 5, 6, 7 and 8. These can be represented as copy-lists as follows:
%
\begin{longtable}{|c|c|c|c|l|}
  \hline
  Node & Outd. & Ref. & Copy List & Extra Nodes \\ \hline
  14 & 9  & 0 & --        & 3, 10, 11, 13, 14, 17, 19, 21, 25 \\
  15 & 11 & 1 & 011011110 & 5, 12, 20, 24, 33 \\ \hline
\end{longtable}
%
The copy-lists can be encoded with RLE and represented as copy-blocks as
follows:
%
\begin{longtable}{|c|c|c|c|c|l|}
  \hline
  Node & Outd. & Ref. & \# & Copy Block & Extra Nodes \\ \hline
  14 & 9  & 0 & -- & --    & 3, 10, 11, 13, 14, 17, 19, 21, 25 \\
  15 & 11 & 1 & 5  & 00103 & 5, 12, 20, 24, 33 \\ \hline
\end{longtable}
