\exercise

Given a dictionary of $2^{16}$ strings,

\begin{itemize}

  \item compute the error rate of a Bloom Filter which uses an array of $2^{20}$
  bits and an optimal number of hash functions \emph{(assume that logs are in
  base 2)};

  \item say when the use of the Bloom Filter is advantageous in space with
  respect to the use of a standard hash table, given that the strings have
  length $L$ bits each and pointers cost 32 bits each.

\end{itemize}

\solution

The optimal number of hash functions is given by
%
\begin{align}
  k = \frac{m}{n}\ln 2 = 2^4 \ln 2
\end{align}
%
while the error rate is given by
%
\begin{align}
  \varepsilon = \left( 1 - e^{-\frac{kn}{m}}\right)^k \approx 0.618^{16}.
\end{align}

Regarding the second point, if we use an hash table then, only for the
initialization (\emph{without inserting any string in the dictionary}), for each
of its slot we have to allocate 2 pointers (one for the string, one for the
bucket), so $64 \cdot 2^d$ bits, where $d$ is the dimension of the dictionary.
For the Bloom Filter, instead, we have to allocate just $2^d$ bits, so the hash
table is alway disadvantageous.
