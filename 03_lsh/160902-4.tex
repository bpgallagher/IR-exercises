\exercise

Given a set of binary vectors,
%
\begin{enumerate}

  \item describe how it is computed an LSH-fingerprint of a binary vector
  \textbf{v} to fast to approximate the Hamming distance between vector-pairs
  with high probability \emph{(Hint: state clearly the parameters involved in
  the setting)};

  \item state and prove how the Hamming distance between two vectors \textbf{v}
  and \textbf{w} is related to the probability of declaring a “match” between
  LSH-fingerprints of \textbf{v} and \textbf{w}, according to the various
  parameters involved;

  \item apply your description on the set of vectors $V = \{00000, 00100,
  01010\}$, and use it to estimate Hamming(00000, 01010) and Hamming(00000,
  00100).

\end{enumerate}

\begin{enumerate}

  \item[] \refstepcounter{enumi}\hskip-\labelwidth\hskip-\labelsep\solution
  \labelenumi\ We compute a sketch of all the vectors of the set via LSH: for
  each vector we take $L$ $k$-projections $h_1$, $h_2$, \dots, $h_L$ and
  generate
  %
  \begin{align*}
    g(\textbf{u}) = \langle h_1(\textbf{u}), h_2(\textbf{u}), \dots,
    h_L(\textbf{u}) \rangle
  \end{align*}
  %
  We know that, for every $i = 1, \dots, L$ and for every $\mathbf{v, w} \in V$
  it holds that
  %
  \begin{align}\label{eq:lsh-p1}
    \mathbb{P}[h_i(\textbf{v}) = h_i(\textbf{w})] =
    \left( 1 - \frac{d_H(\textbf{v}, \textbf{w})}{n} \right)^k
  \end{align}
  %
  where $n = |\textbf{v}| = |\textbf{w}|$ and $d_H(\textbf{v}, \textbf{w})$ is
  the Hamming distance between $\textbf{v}$ and $\textbf{w}$. We can make an
  empirical estimation of the previous probability with
  %
  \begin{align}\label{eq:lsh-p2}
    \mathbb{P}[h_i(\textbf{v}) = h_i(\textbf{w})] \approx
    1 - \frac{d_H(g(\textbf{v}), g(\textbf{w}))}{L}.
  \end{align}
  %
  We can then combine \autoref{eq:lsh-p1} and \ref{eq:lsh-p2} and get the
  Hamming distance:
  %
  \begin{align*}
    d_H(\textbf{v}, \textbf{w}) &= n \cdot \left( 1 -
    \sqrt[k]{\mathbb{P}[h_i(\textbf{v}) = h_i(\textbf{w})]}\right) \\
    &\approx n \cdot \left( 1 - \sqrt[k]{1 - \frac{d_H(g(\textbf{v}),
    g(\textbf{w}))}{L}}\right)
  \end{align*}

  \item Two fingerprints match ($g(\textbf{v}) \simeq g(\textbf{w})$) if they
  share at least a component in the same position, and the relation between the
  probability of a match and the Hamming distance is
  %
  \begin{align*}
    \mathbb{P}[g(\textbf{v}) \simeq g(\textbf{w})] = 1 - \left( 1 - \left( 1 -
    \frac{d_H(\textbf{v}, \textbf{w})}{n} \right)^k \right)^L
  \end{align*}

  \paragraph{Proof} The probability that, given two vectors $\textbf{v}$ and
  $\textbf{w}$, they share the same bit at the same index $j$ is
  %
  \begin{align*}
    \mathbb{P}[\textbf{v}_j = \textbf{w}_j] = 1 - \frac{d_H(\textbf{v},
    \textbf{w})}{n}.
  \end{align*}
  %
  The probability that the same two vectors share $k$ positions is
  %
  \begin{align*}
    \mathbb{P}[\textbf{v}_{i_1} = \textbf{w}_{i_1} \wedge \dots \wedge
    \textbf{v}_{i_k} = \textbf{w}_{i_k}]
    &= \mathbb{P}[h_i(\textbf{v}) = h_i(\textbf{w})] \\
    &= \left( \mathbb{P}[\textbf{v}_j = \textbf{w}_j] \right)^k \\
    &= \left( 1 - \frac{d_H(\textbf{v}, \textbf{w})}{n} \right)^k.
  \end{align*}
  %
  The probability that at least 1 bit is different is
  %
  \begin{align*}
    \mathbb{P}[\textbf{v}_{i_1} \neq \textbf{w}_{i_1} \vee \dots \vee
    \textbf{v}_{i_k} \neq \textbf{w}_{i_k}]
    &= \mathbb{P}[h_i(\textbf{v}) \neq h_i(\textbf{w})] \\
    &= 1 - \mathbb{P}[h_i(\textbf{v}) = h_i(\textbf{w})] \\
    &= 1 - \left( 1 - \frac{d_H(\textbf{v}, \textbf{w})}{n} \right)^k.
  \end{align*}
  %
  The probability that this will happen for all $L$ projections is
  %
  \begin{align*}
    \mathbb{P}[h_1(\textbf{v}) \neq h_1(\textbf{w}) \wedge \dots \wedge
    h_L(\textbf{v}) \neq h_L(\textbf{w})] = &\left(\mathbb{P}[h_i(\textbf{v})
    \neq h_i(\textbf{w})] \right)^L \\ = &\left( 1 - \left( 1 -
    \frac{d_H(\textbf{v}, \textbf{w})}{n} \right)^k \right)^L.
  \end{align*}
  %
  Finally, the probability that at least a projection is shared is
  %
  \begin{align*}
    \mathbb{P}[g(\textbf{v}) \simeq g(\textbf{w})] &=
    \mathbb{P}[h_1(\textbf{v}) = h_1(\textbf{w}) \vee \dots \vee
    h_L(\textbf{v}) = h_L(\textbf{w})] \\ &= 1 -\mathbb{P}[h_1(\textbf{v}) \neq
    h_1(\textbf{w}) \wedge \dots \wedge h_L(\textbf{v}) \neq h_L(\textbf{w})] \\
    &= 1 - \left(\mathbb{P}[h_i(\textbf{v}) \neq h_i(\textbf{w})] \right)^L \\
    &= 1 - \left( 1 -\left( 1 - \frac{d_H(\textbf{v}, \textbf{w})}{n} \right)^k
    \right)^L.
  \end{align*}

  \item Given $V = \{00000, 00100, 01010\}$, we use LSH with parameters $L = 2$
  and $k = 2$. We choose the projections $I_1 = \{ 0, 2 \}$, $I_2 = \{ 3, 4
  \}$ and $I_3 = \{1, 3 \}$ and we obtain the following sketches:
  %
  \begin{table}[H]
    \centering
    \begin{tabular}{c|c|c|c|}
      & $I_1$ & $I_2$ & $I_3$ \\ \hline
      $\mathbf{v_1}$ & 00 & 00 & 00\\ \hline
      $\mathbf{v_2}$ & 01 & 00 & 00\\ \hline
      $\mathbf{v_3}$ & 00 & 10 & 11\\ \hline
    \end{tabular}
  \end{table}
  %
  Then, we compute the Hamming distances between the sketches:
  %
  \begin{table}[H]
    \centering
    \begin{tabular}{c|c|c|c|}
      & $g(\mathbf{v_1})$ & $g(\mathbf{v_2})$ & $g(\mathbf{v_3})$ \\ \hline
      $g(\mathbf{v_1})$ & 0 & 1 & 2 \\ \hline
      $g(\mathbf{v_2})$ & 1 & 0 & 3 \\ \hline
      $g(\mathbf{v_3})$ & 2 & 3 & 0 \\ \hline
    \end{tabular}
  \end{table}
  %
  We can now estimate the Hamming distances between the original vectors:
  %
  \begin{align*}
    d_H(\mathbf{v}_1, \mathbf{v}_2) &\approx
    n \cdot \left( 1 - \sqrt[k]{1 - \frac{d_H(g(\mathbf{v}_1),
    g(\mathbf{v}_2)}{L}} \right) \\
    &= 5 \cdot \left( 1 - \sqrt{1 - \frac{1}{3}} \right) \approx 0.918
    \approx 1 \\
    d_H(\mathbf{v}_1, \mathbf{v}_3) &\approx
    n \cdot \left( 1 - \sqrt[k]{1 - \frac{d_H(g(\mathbf{v}_1),
    g(\mathbf{v}_3)}{L}} \right) \\
    &= 5 \cdot \left( 1 - \sqrt{1 - \frac{2}{3}} \right) \approx 2.113
    \approx 2 \\
    d_H(\mathbf{v}_2, \mathbf{v}_3) &\approx
    n \cdot \left( 1 - \sqrt[k]{1 - \frac{d_H(g(\mathbf{v}_2),
    g(\mathbf{v}_3)}{L}} \right) \\
    &= 5 \cdot \left( 1 - \sqrt{1 - \frac{3}{3}} \right) = 5\ (\neq 3) \\
  \end{align*}

\end{enumerate}
