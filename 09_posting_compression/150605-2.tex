\exercise

Given the sequence $S = (1, 2, 1, 1, 4, 7, 7, 4, 1, 3, 9)$, show:
%
\begin{itemize}
  \item the PForDelta encoding with b=2 and base=0,
  \item the Elias-Fano encoding (warning: remind that EF encodes monotonic
    sequences).
\end{itemize}

\solution

In PForDelta we encode values in the interval $[base,base+2^b-1]$ moving
them to the interval $[0, 2^b-1]$ and storing their representation in $b$ bits.
Values outside this range are marked with an escape sequence
$\triangleright=00$, and encoded in a separate list $L$.

In the given case, since $base=0$, we do not translate the items, but we store
their binary representations
%
$$01\;10\;01\;01\;\triangleright\;\triangleright\;\triangleright\;\triangleright\;01\;11\;\triangleright,$$
%
with the exceptions list
$$L=(4,7,7,4,9).$$

To encode $S$ with Elias-Fano, we create a monotonic sequence decompressing $S$
with reverse gap encoding:

$$S'=(1,3,4,5,9,16,23,27,28,31,40).$$

Then, in order to encode the values obtained as binary strings, we use $b = \lceil \log_2{40}\rceil = 6$ bit.
$w = \lceil log_2{\frac{2^b}{n}}\rceil = 3$ of this bits are used to produce the $L$ sequence, and the remaining ones
are used to produce the $H$ sequence\footnote{Refer to the note about integers' encoding given during the course in order
to better understand how sequence $H$ is constructed.}.

\begin{center}
  \begin{tabular}{ c | c | c }
    $S'$ & $z$ & $w$ \\ \hline
     1 & 000 & 001 \\
     3 & 000 & 011 \\
     4 & 000 & 100 \\
     5 & 000 & 101 \\
     9 & 001 & 001 \\
     16 & 010 & 000 \\
     23 & 010 & 111 \\
     27 & 010 & 011 \\
     28 & 011 & 011 \\
     31 & 011 & 100 \\
     40 & 011 & 111
  \end{tabular}
\end{center}

\begin{align*}
    &L = 001011100101001000111011100111000, \\
    &H = 1111010110111001000.
\end{align*}
