\exercise

You are given the two files: $f_{old} = \mathtt{"cane gatto orso"}$, $f_{new} =
\mathtt{"cane matto dorso"}$, and assume a block size $B = 3$ chars.
%
\begin{itemize}

  \item Show the execution of the algorithm rsync.
  \emph{(comment the various steps)}

  \item Show the execution of the algorithm zsync.
  \emph{(comment the various steps)}

\end{itemize}

\solution

\begin{description}

  \item[rsync] Client starts computing the hashes of $f_{old}$ divided by blocks
  of 3 chars:
  %
  \begin{align*}
    \underbracket{\vphantom{\tt g}\texttt{c a n}}_{H_1}\
    \underbracket{\texttt{e \_ g}}_{H_2}\
    \underbracket{\vphantom{\tt g}\texttt{a t t}}_{H_3}\
    \underbracket{\vphantom{\tt g}\texttt{o \_ o}}_{H_4}\
    \underbracket{\vphantom{\tt g}\texttt{r s o}}_{H_5}
  \end{align*}
  %
  The client sends the hashcodes to the server, which compares them to the ones
  produced by the rolling hash on $f_{new}$:
  %
  \begin{align*}
    \underbracket{\vphantom{\tt g}\texttt{c a n}}_{H_1}\ \texttt{e \_ m}\
    \underbracket{\vphantom{\tt g}\texttt{a t t}}_{H_3}\ \texttt{o \_ d o}\
    \underbracket{\vphantom{\tt g}\texttt{r s o}}_{H_5}
  \end{align*}
  %
  The server then sends: $H_1$, "\texttt{e g}", $H_3$, "\texttt{o do}", $H_5$.

  \item[zsync] Server starts computing the hashes of $f_{new}$ divided block by
  block:
  %
  \begin{align*}
    \underbracket{\vphantom{\tt g}\texttt{c a n}}_{H_1}\
    \underbracket{\vphantom{\tt g}\texttt{e \_ m}}_{H_2}\
    \underbracket{\vphantom{\tt g}\texttt{a t t}}_{H_3}\
    \underbracket{\vphantom{\tt g}\texttt{o \_ d}}_{H_4}\
    \underbracket{\vphantom{\tt g}\texttt{o r s}}_{H_5}\
    \underbracket{\vphantom{\tt g}\texttt{o \$}}_{H_6}\
  \end{align*}
  %
  The server sends the hashcodes to the client, which compares them to the ones
  produced by the rolling hash on $f_{old}$:
  %
  \begin{align*}
    \underbracket{\vphantom{\tt g}\texttt{c a n}}_{H_1}\
    \texttt{e \_ g}\
    \underbracket{\vphantom{\tt g}\texttt{a t t}}_{H_3}\
    \texttt{o \_}\
    \underbracket{\vphantom{\tt g}\texttt{o r s}}_{H_5}\
    \underbracket{\vphantom{\tt g}\texttt{o \$}}_{H_6}\
  \end{align*}
  %
  Then client sends the vector: 101011. The server then computes the \emph{gzip}
  of the unmatched substrings given the matched ones:
  %
  \begin{table}[H]
    \centering
    \begin{tabular}{r|lcl}
    \tt{c a n a t t o r s o \$} & \colorbox{yellow}{\tt e}
    {\tt \_ m o \_ d \$} & &
    $\langle 0, 0, \text{\tt e} \rangle$ \\
    \tt{c a n a t t o r s o \$} & {\tt e} \colorbox{yellow}{\tt \_}
    {\tt m o \_ d \$} & & $\langle 0, 0, \text{\tt \_} \rangle$ \\
    \tt{c a n a t t o r s o \$} & {\tt e \_} \colorbox{yellow}{\tt m}
    {\tt o \_ d \$} & & $\langle 0, 0, \text{\tt m} \rangle$ \\
    \tt{c a n a t t o r s} \colorbox{pink}{o}{\tt \$}& {\tt e \_ m}
    \colorbox{yellow}{\tt o \_} {\tt d \$} & &
    $\langle 5, 1, \text{\tt \_} \rangle$ \\
    \tt{c a n a t t o r s o \$} & {\tt e \_ m o \_} \colorbox{yellow}{\tt d}
    {\tt \$} & & $\langle 0, 0, \text{\tt d} \rangle$ \\
    \tt{c a n a t t o r s o \$} & {\tt e \_ m o \_ d} \colorbox{yellow}{\tt \$}
    & & $\langle 0, 0, \text{\tt \$} \rangle$ \\
    \end{tabular}
  \end{table}

\end{description}
